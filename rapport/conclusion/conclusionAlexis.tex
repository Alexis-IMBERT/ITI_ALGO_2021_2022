Ce projet de corrceteur orthographique m'a permis de découvrir la conception et le dévellopement d'un logiciel informatique par le biai du language C.
J'avais déjà étudier somairement le C en premier année post BAC mais je mettais arrété à des programmes n'ayant qu'un fichier 
(ou alors plusieurs fichier mais avec un dévellopement guidé et un makefile donné).
J'ai aussi apris l'utilisation de l'environnement git, l'utilisation poussé du language \LaTeX, la conception de documentation grace à doxygen
ou encore l'utilisation de d'outils comme Valgrind ou l'option de compilation -fsainitize=address pour détecter les fuites de mémoires. 
Il m'a aussi permis de découvrir la gestion d'équipe et un peu de pédagogie pour expliquer le fonctionnement de certain outils à  mes collègues. 
J'ai aussi confirmer confirmer des compétences apris plus tot en TP : la réalisation d'un makefile, l'utilisation du framework CUnit.\\

Nous avons eu du mal à nous lancer au début du projet, l'apprentissage de l'environnement git, de la gestion du dépot et la mise en place du rapport, 
pour m'a part ont pris du temps sur l'analyse.\\

Une fois le dévellopement lancé le plus dur pour moi a été de coordonner tout le monde pour que les tests soit prêt avant que la personne implémente sa partie.
Dans mon cas je me suis retrouvé à déveloper sans tests et il a été très dur de répérer les erreurs avant de les avoirs.\\

La deuxième difficulté a été l'optimisation du programme : une fois terminer ma partie terminé et les tests unitaires validé 
j'ai lancé le programme en ajoutant les 300 000 mots. Je n'ai pas vu le programme s'arrété et je l'ai arrété avant en mettant 
un affichage j'ajoutais 100 000 mots en 5 min ce qui n'était pas acceptable.