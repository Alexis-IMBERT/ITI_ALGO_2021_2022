Ce projet de correcteur orthographique m'a permis de découvrir la conception et le développement d'un logiciel informatique par le biais du langage C.
J'avais déjà étudié sommairement le C en première année post BAC mais je m'étais arrêté à des programmes n'ayant qu'un fichier 
(ou alors plusieurs fichiers mais avec un développement guidé et un makefile donné).
J'ai aussi appris l'utilisation de l'environnement GIT, l'utilisation poussé du langage \LaTeX, la conception de documentation grâce à Doxygen.
D'autres outils m'ont aussi aidé lors du débuggage comme Valgrind ou l'option de compilation --fsainitize=address pour détecter les fuites de mémoires. 
Ce projet m'a également permis de découvrir la gestion d'équipe ainsi que la pédagogie pour expliquer le fonctionnement de certain outils à mes collègues. 
J'ai aussi confirmé des compétences apprises plus tôt en TP comme la réalisation d'un makefile, l'utilisation du framework CUnit.\\

Néanmoins, nous avons eu du mal à nous lancer au début du projet.
L'apprentissage de l'environnement GIT, de la gestion du dépôt et la mise en place du rapport ont étés très chronophages pour moi.\\

Une fois le développement lancé, la partie coordination que j'ai prise en charge en temps que chef d'équipe a été complexe mais riche d'apprentissage.

La deuxième difficulté a été l'optimisation du programme. 
Une fois ma partie terminée et les tests unitaires validés, j'ai lancé le programme en ajoutant les 300 000 mots à la fois.
Le temps d’exécution étant beaucoup trop long je suis passé à 100 000, toujours trop long.
Pour résoudre ce problème j'ai du repensé la manière d'obtenir la hauteur de l'arbre en la stockant dans chaque nœud au lieux de la calculer à chaque fois.
