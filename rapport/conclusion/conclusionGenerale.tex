\subsubsection{Répartition des taches}
Pour ce qui est de la répartition des tâches, nous avions essayé de distribuer les différents TADs à chacun de façon équitable, de même pour ce qui est du développement et des tests unitaires. Néanmoins, certaines parties étaient plus dures à gérer que d'autres, impliquant donc un travail collectif sur celles-ci comme vous pouvez le voir dans les tableaux ci-dessous. De même, cela n'apparaît pas dans ceux-ci mais de l'aide entre membres du groupe a parfois été réquisitionnée et nous a permis d'avancer plus facilement.
% Please add the following required packages to your document preamble:
% \usepackage{longtable}
% Note: It may be necessary to compile the document several times to get a multi-page table to line up properly
\begin{longtable}[c]{|r|c|c|c|c|c|}
	\hline
	\multicolumn{1}{|c|}{} &
	  \textbf{Analyse} &
	  \textbf{\begin{tabular}[c]{@{}c@{}}Conception \\ Préliminaire\end{tabular}} &
	  \textbf{\begin{tabular}[c]{@{}c@{}}Conception \\ Détaillé\end{tabular}} &
	  \textbf{Dévellopement} &
	  \textbf{Test Unitaire} \\ \hline
	\endhead
	%
	\textbf{TAD Fichier Texte} &
	  x &
	  Alexis &
	  x &
	  x &
	  x \\ \hline
	\textbf{TAD Mot} &
	  Léo \& Amina &
	  Amina &
	  Amina &
	  Léo &
	  Ruth \\ \hline
	\textbf{TAD Dictionnaire} &
	  Léo \& Amina &
	  Ruth &
	  Léo &
	  Alexis &
	  \begin{tabular}[c]{@{}c@{}}Amina \& Léo\\ \& Ruth \& Alexis\end{tabular} \\ \hline
	\textbf{\begin{tabular}[c]{@{}r@{}}TAD Correcteur \\ Orthographique\end{tabular}} &
	  Léo \& Amina &
	  Amina &
	  Léo &
	  Ruth &
	  Léo \\ \hline
\end{longtable}

	
% Please add the following required packages to your document preamble:
% \usepackage{multirow}
% \usepackage{longtable}
% Note: It may be necessary to compile the document several times to get a multi-page table to line up properly
\begin{longtable}[c]{|ll|}
	\hline
	\multicolumn{2}{|c|}{\textbf{Autres Taches :}} \\ \hline
	\endfirsthead
	%
	\endhead
	%
	\multicolumn{1}{|l|}{\textbf{\begin{tabular}[c]{@{}l@{}}Rédaction de l'analyse \\ et de la conception détaillée \\ dans le rapport\end{tabular}}} &
	  \begin{tabular}[c]{@{}l@{}}Alexis , pour le TAD Fichier Texte\\ Ruth pour les autres\end{tabular} \\ \hline
	\multicolumn{1}{|l|}{\textbf{Gestion du git}} &
	  Alexis \\ \hline
	\multicolumn{1}{|l|}{\multirow{2}{*}{\textbf{Gestion rapport}}} &
	  Alexis : structure et arborescence \\ \cline{2-2} 
	\multicolumn{1}{|l|}{} &
	  \begin{tabular}[c]{@{}l@{}}Ruth Contenue de l'analyse,\\ La correction des fautes d'orthographes\\ La réalisation Makefile\end{tabular} \\ \hline
	\multicolumn{1}{|l|}{\textbf{L'Analyse descendante}} &
	  Léo \\ \hline
	\multicolumn{1}{|l|}{\textbf{\begin{tabular}[c]{@{}l@{}}Makefile du programme \\ + script de génération, rapport, \\ programme et documentation\end{tabular}}} &
	  Alexis \\ \hline
	\multicolumn{1}{|l|}{\textbf{\begin{tabular}[c]{@{}l@{}}Développement de charger \\ et enregistrer dictionnaire \\ en version non naïve\end{tabular}}} &
	  Léo \\ \hline
	\multicolumn{1}{|l|}{\textbf{\begin{tabular}[c]{@{}l@{}}Recherche et corrections \\ des fuites de mémoires\end{tabular}}} &
	  Léo \\ \hline
	\multicolumn{2}{|c|}{\textbf{Test dictionnaire}} \\ \hline
	\multicolumn{1}{|l|}{\textbf{Les rotations (simples et doubles)}} &
	  Léo \\ \hline
	\multicolumn{1}{|l|}{\textbf{hauteur d'un arbre}} &
	  Alexis \\ \hline
	\multicolumn{1}{|l|}{\textbf{Rééquilibrer}} &
	  Léo \\ \hline
	\multicolumn{1}{|l|}{\textbf{estPrésent}} &
	  Amina \\ \hline
	\multicolumn{1}{|l|}{\textbf{ajouterMot}} &
	  Alexis \\ \hline
	\multicolumn{1}{|l|}{\textbf{\begin{tabular}[c]{@{}l@{}}chargerDictionnaire \\ et EnregistrerDictionnaire\end{tabular}}} &
	  Ruth \\ \hline
	\multicolumn{1}{|l|}{\textbf{\begin{tabular}[c]{@{}l@{}}Création de la fonction \\ Dictionnaire\_SontEgaux\end{tabular}}} &
	  Alexis \\ \hline
	\end{longtable}

\subsubsection{Conclusion}
Ce projet s'est globalement bien passé, malgré quelques difficultés que nous avons su surmonter et qui nous ont beaucoup appris nous permettant ainsi de progresser.
Nous avons développé notre solidarité et notre communication.

Au final, le projet fonctionne, nous avons eu le temps d'implémenter toutes les fonctions demandées dans le sujet. 
Nous avons pu régler les quelques problèmes de fuite de mémoires.

Toutefois, nous avons pensé à quelques améliorations suivantes:
\begin{itemize}
\item Afin de ne pas limiter la hauteur théorique de l'arbre nous pourrions stocker la différence de hauteur des 2 fils au lieu de la hauteur elle-même qui pourrait théoriquement nous limiter.
\item Améliorer la prise en charge des accents.
\item Améliorer la prise en compte de certains caractères comme l'apostrophe.
\end{itemize}

Avec plus de temps nous aurions été ravis de les effectuer. Nous aurions pu aussi découvrir le principe d'amélioration continue.
