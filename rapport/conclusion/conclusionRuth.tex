En tant que conclusion personnelle, je dirais que ce projet a probablement été un des premiers retranscrivant le plus la réalité de la vie professionnelle en tant qu'ingénieur.\par
En effet, ce fut la première fois qu'un chef de projet fut là pour organiser tout le fonctionnement du projet, ainsi que l'utilisation intensive du GIT. Néanmoins, j'avoue que le langage de programmation utilisé fut assez compliqué à apprendre et comprendre notamment sur l'utilisation des pointeurs. C'est pourquoi, malgré le temps conséquent qu'il nous a demandé, ce projet nous a permis d'acquérir de nouvelles connaissances. J'ai aussi pu apprendre à manipuler de nombreux outils particulièrement utiles comme Valgrind, ou CUnit plus en particulier. Ces deux outils ont largement facilité les étapes de débuggage du programme.\par
Globalement, malgré la difficulté du projet, ce fut une bonne expérience notamment au niveau travail collectif ainsi qu'au niveau de l'apprentissage de nouvelles choses essentielles pour notre évolution personnelle en tant futurs ingénieurs dans l'informatique.
